\documentclass[usenames,dvipsnames,aspectratio=169]{beamer}
\usepackage{../common/prg}

\title[5. előadás]{Programozás}
\subtitle{(GKxB\_INTM114)}

\begin{document}

%1
\begin{frame}[plain]
  \titlepage
  \logoalul
\end{frame}

%2
\section{Logikailag összetartozó adatok kezelése}
\subsection{Azonos elemszámú, eltérő típusú/rendeltetésű tömbök használata}
\begin{frame}
  \begin{exampleblock}{\textattachfile{zh1.cpp}{zh1.cpp} Legeredményesebb hallgatók kikeresése}
    \scriptsize
    \lstinputlisting[style=cpp,linerange={1-16},numbers=left,firstnumber=1]{zh1.cpp}
  \end{exampleblock}
\end{frame}

%3
\begin{frame}
  \begin{exampleblock}{\textattachfile{zh1.cpp}{zh1.cpp}}
    \scriptsize
    \lstinputlisting[style=cpp,linerange={17-25},numbers=left,firstnumber=17]{zh1.cpp}
  \end{exampleblock}
  \vfill
  Probléma:
  \begin{itemize}
    \item egy hallgató neve és pontszáma szorosabb logikai kapcsolatban van, mint pl. a különféle hallgatók nevei
    \item a tömbök viszont ezt nem tükrözik
  \end{itemize}
\end{frame}

%4
\subsection{Struktúrák fő jellemzői}
\begin{frame}
  Struktúrák fő jellemzői:
  \begin{itemize}
    \item logikailag összetartozó változók csoportjának egyszerűbb kezelése
    \item összetett, felhasználói adattípus hozható létre
    \item egy vagy több, elnevezett \emph{tag} együttese, önálló azonosítóval
    \item lehetőségek:
    \begin{itemize}
      \item hozzárendelés (másolás)
      \item átadhatók függvénynek paraméterként
      \item lehet függvény visszatérési értéke
    \end{itemize}
    \item \kiemel{nem lehetséges: összehasonlítás} (esetleg tagonként)
    \item struktúratag \emph{szinte} bármiből lehet
  \end{itemize}
\end{frame}

%5
\begin{frame}
  \begin{exampleblock}{\textattachfile{zh2.cpp}{zh2.cpp} Legeredményesebb hallgatók kikeresése}
    \vspace{-.2cm}
    \meret{9}
    \lstinputlisting[style=cpp,linerange={1-15},numbers=left,firstnumber=1]{zh2.cpp}
    \vspace{-.2cm}
  \end{exampleblock}
\end{frame}


%6
\begin{frame}
  \begin{exampleblock}{\textattachfile{zh2.cpp}{zh2.cpp}}
    \meret{9}
    \lstinputlisting[style=cpp,linerange={16-28},numbers=left,firstnumber=16]{zh2.cpp}
  \end{exampleblock}
\end{frame}

%7
\subsection{Struktúradeklaráció}
\begin{frame}[fragile]
  \footnotesize
  Általános alak: \textbf{struct} \emph{{<}struktúracímke>} \emph{{<}{struktúratag-deklarációlista}> <azonosítólista>};
  \begin{exampleblock}{Struktúradeklaráció példa}
    \vspace{-.2cm}
    \begin{verbatim}
struct hallgato { // Struktúra deklarálása
  string nev;
  int pontszam;
};

struct hallgato mari; // Változók definiálása
hallgato karcsi, hg[1000];
\end{verbatim}
  \vspace{-.2cm}
  \end{exampleblock}
  \begin{itemize}
    \item \texttt{hallgato} a struktúra címkéje, a típust azonosítja:\\
      \texttt{struct hallgato mari;} \\
      \texttt{hallgato karcsi;}
    \item Tagok: \texttt{nev, pontszam} (egyedi azonosítók)
    \item Változók: \texttt{mari}, \texttt{karcsi}\\ 
      \texttt{hg[1000]} egy 1000 elemű struktúratömb
  \end{itemize}
\end{frame}

%8
\begin{frame}[fragile]
  Hol deklaráljuk a struktúrát?
  \begin{itemize}
    \item a típus első felhasználása előtt
    \item jellemzően a forrás elején, minden függvényen kívül
  \end{itemize}
  Minden deklaráció \emph{egyedi típust hoz létre}, melyek nem azonosak akkor sem, ha szerkezetük megegyezik.
  \meret{7}
  \begin{exampleblock}{Típusok különbözősége}
    \vspace{-.2cm}
    \begin{verbatim}
struct hallgato1 {
  string nev; int pontszam;
};

struct hallgato2 {
  string nev; int pontszam;
};

hallgato1 mari;
hallgato2 gizi;
gizi = mari; // error: no match for 'operator=' 
             // (operand types are 'main()::hallgato2' and 'main()::hallgato1')
\end{verbatim}
    \vspace{-.2cm}
  \end{exampleblock}
\end{frame}

%9
\begin{frame}[fragile]
  \begin{compactitem}
    \item Struktúratag lehet pl.
    \begin{compactitem}
      \small
      \item korábban definiált szerkezetű struktúra
      \item beágyazott struktúra, akár címke nélkül is
      \item tömb
      \item (függvény $\to$ következő félév anyaga)
    \end{compactitem}
    \item A tag azonosítójának csak a struktúrán belül kell egyedinek lennie
    \item A deklaráció végén lévő pontosvessző nem hagyható el!
  \end{compactitem}
  \small
  \begin{exampleblock}{Helyes tagdeklarációk}
    \vspace{-.2cm}
    \begin{verbatim}
struct s { int i; };
struct tag_dekl {
  struct s s1;
  struct { int i; long l; } b;
  int szamok[30];
};
\end{verbatim}
    \vspace{-.2cm}
  \end{exampleblock}
\end{frame}

%10
\begin{frame}[fragile]
  Struktúratag típusa nem lehet pl.
  \begin{itemize}
    \item \texttt{void}
    \item saját maga
  \end{itemize}
  \begin{alertblock}{Hibás tagdeklarációk}
    \vspace{-.2cm}
    \begin{verbatim}
struct nem_teljes;
struct tag_hiba {
  void v; /* error: variable or field 'v' declared void */
  struct nem_teljes s; /* error: field 's' has incomplete type */
  struct tag_hiba th; /* error: field 'th' has incomplete type */
};
\end{verbatim}
    \vspace{-.2cm}
  \end{alertblock}
\end{frame}

%11
\subsection{Struktúratagok elérése}
\begin{frame}[fragile]
  Tagelérés operátor / szelekciós operátor / tagszelektor
  \begin{compactitem}
    \item \emph{struktúra\kiemel{.}tag}
    \item Magas prioritású operátor, balról jobbra köt
  \end{compactitem}
  \scriptsize
  \begin{exampleblock}{Struktúratagok elérése, értékadások}
    \vspace{-.2cm}
    \begin{verbatim}
struct hallgato {
  string nev;
  string telSzamok[2];
  struct {
    int ev, ho, nap;
  } szulDatum;
};
/* ... */
hallgato gizi;
gizi.nev = "Kovács Gizella";
gizi.telSzamok[0] = "+36 1 123-4567"; gizi.telSzamok[1] = "96/123-456";
gizi.szulDatum.ev = 1990; gizi.szulDatum.ho = 1; gizi.szulDatum.nap = 2;
\end{verbatim}
    \vspace{-.2cm}
  \end{exampleblock}
\end{frame}

%12
\subsection{Struktúrák inicializálása}
\begin{frame}[fragile]
  Az inicializáció során a tagok a deklarációbeli sorrendjükben veszik fel az inicializátorlista elemeinek értékét.\\
  Az inicializátor azonos típusú struktúra is lehet.
  \begin{exampleblock}{Struktúra inicializálása}
    \vspace{-.2cm}
    \begin{verbatim}
struct hallgato {
  string nev, neptun;
  int ev, ho, nap;
}; 

hallgato gizi = { "Kovács Gizella", "A1B2C3", 1990, 4, 23 };
hallgato mari = gizi;
// mari = { "Nagy Maria", "ABC123", 1995, 5, 6 };
// error: 'mari' does not name a type
\end{verbatim}
    \vspace{-.2cm}
  \end{exampleblock}
\end{frame}

%13
\begin{frame}[fragile]
  \small
  Struktúrába ágyazott tagok inicializálása: beágyazott inicializátorokkal
  \begin{exampleblock}{Beágyazott struktúra és tömb inicializálása}
    \vspace{-.2cm}
    \scriptsize
    \begin{verbatim}
struct datum {
  int ev, ho, nap;
};
struct hallgato {
  string nev, neptun;
  string telSzamok[2];
  datum szulDatum, diplomaSzerzes;
};
hallgato gizi = { "Kovács Gizella", "A1B2C3", 
  {"+36 1 123-4567", "+36 20 987-6543"}, 
  {1990, 4, 23}, {2015, 6, 3} };
\end{verbatim}
    \vspace{-.2cm}
  \end{exampleblock}
  \begin{compactitem}
    \item Inicializátorlista elemszáma nem haladhatja meg a tagok számát!
    \item Ha viszont kevesebb elemű $\to$ nullázás
    \item Aggregátumok esetén a \texttt{\{ \}} elhagyhatók, ill. valamennyi inicializátor köré is helyezhető, de
\emph{célszerű} követni az aggregátum szerkezetét
  \end{compactitem}
\end{frame}

%14
\begin{frame}[fragile]
  Kijelölt inicializátorok (\hiv{\href{https://www.cppstories.com/2021/designated-init-cpp20/}{designated initializers}}, C++20) használata: közvetlen hivatkozás a tagokra
  \begin{exampleblock}{Dátum struktúra inicializálása}
    \footnotesize
    \vspace{-.2cm}
    \begin{verbatim}
struct datum {
  int ev =  { 2023 };
  int ho =  {   01 };
  int nap = {   01 };
} szulDatum = { .ho = 1, .nap = 26 };
\end{verbatim}
    \vspace{-.2cm}
  \end{exampleblock}
  \footnotesize
  \begin{columns}[T]
    \column{.45\textwidth}
      \begin{compactitem}
        \item Csak aggregátumok inicializálására használhatóak
        \item (Csak nem statikus tagok inicializálhatók)
        \item Az inicializálók sorrendjének egyeznie kell a tagok sorrendjével
      \end{compactitem}
    \column{.45\textwidth}
    \begin{compactitem}
      \item Nem kötelező az összes tagot inicializálni
      \item Nem keverhető a hagyományos inicializáció a kijelölt inicializálókkal
      \item Egy tag egy inicializálóval inicializálható
      \item Nem lehet az inicializátorokat egymásba ágyazni
    \end{compactitem}
  \end{columns}
\end{frame}

%15
\subsection{Mintapéldák -- Dátumok kezelése}
\begin{frame}
  \begin{exampleblock}{\textattachfile{naptar1.cpp}{naptar1.cpp} Dátumkezelő függvények}
    \scriptsize
    \vspace{-.2cm}
    \lstinputlisting[style=cpp,linerange={4-20},numbers=left,firstnumber=4]{naptar1.cpp}
    \vspace{-.2cm}
  \end{exampleblock}
\end{frame}

%16
\begin{frame}
  \begin{exampleblock}{\textattachfile{naptar1.cpp}{naptar1.cpp}}
    \footnotesize
    \lstinputlisting[style=cpp,linerange={22-35},numbers=left,firstnumber=22]{naptar1.cpp}
  \end{exampleblock}
\end{frame}

%17
\begin{frame}
  \begin{exampleblock}{\textattachfile{naptar1.cpp}{naptar1.cpp}}
    \lstinputlisting[style=cpp,linerange={37-47},numbers=left,firstnumber=37]{naptar1.cpp}
  \end{exampleblock}
\end{frame}

%18
\begin{frame}
  \begin{exampleblock}{\textattachfile{naptar1.cpp}{naptar1.cpp}}
    \scriptsize
    \lstinputlisting[style=cpp,linerange={49-64},numbers=left,firstnumber=49]{naptar1.cpp}
  \end{exampleblock}
\end{frame}

%19
\begin{frame}
  \begin{exampleblock}{\textattachfile{naptar1.cpp}{naptar1.cpp}}
    \lstinputlisting[style=cpp,linerange={66-74},numbers=left,firstnumber=66]{naptar1.cpp}
  \end{exampleblock}
\end{frame}

%20
\begin{frame}
  \begin{exampleblock}{\textattachfile{naptar1.cpp}{naptar1.cpp}}
    \footnotesize
    \lstinputlisting[style=cpp,linerange={76-89},numbers=left,firstnumber=76]{naptar1.cpp}
  \end{exampleblock}
\end{frame}

%21
\begin{frame}[fragile]
  \begin{block}{Kimenet}
    \vspace{-.2cm}
    \begin{verbatim}
A megadott datum helyes.
2024.3.20 az ev 80. napja, szerda.
Hany nap van karacsonyig? 279
2024 300. napja: 10.26
\end{verbatim}
    \vspace{-.2cm}
  \end{block}
\end{frame}

%22
\subsection{Mintapéldák -- Téglalapok rajzolása}
\begin{frame}[fragile]
  \scriptsize
  \begin{block}{Kimenet (1/2) -- Téglalapok rajzolása}
    \vspace{-.3cm}
    \begin{verbatim}
Rajzprogram - adja meg a téglalapok adatait!
1. teglalap BF sarok X: [0, 78] (negativra vege) 1
1. teglalap BF sarok Y[0, 23] 1
1. teglalap JA sarok X[2, 79] 11
1. teglalap JA sarok Y[2, 24] 11
1. teglalap rajzoló karaktere: |
2. teglalap BF sarok X: [0, 78] (negativra vege) 6
2. teglalap BF sarok Y[0, 23] 6
2. teglalap JA sarok X[7, 79] 16
2. teglalap JA sarok Y[7, 24] 16
2. teglalap rajzoló karaktere: +
3. teglalap BF sarok X: [0, 78] (negativra vege) 15
3. teglalap BF sarok Y[0, 23] 2
3. teglalap JA sarok X[16, 79] 30
3. teglalap JA sarok Y[3, 24] 7
3. teglalap rajzoló karaktere: -
4. teglalap BF sarok X: [0, 78] (negativra vege) -1
...
\end{verbatim}
    \vspace{-.3cm}
  \end{block}
\end{frame}

%23
\begin{frame}[fragile]
  \scriptsize
  \begin{block}{Kimenet (2/2)}
    \vspace{-.3cm}
    \begin{verbatim}
...
 |||||||||||                                                                   
 |||||||||||   ----------------                                                
 |||||||||||   ----------------                                                
 |||||||||||   ----------------                                                
 |||||||||||   ----------------                                                
 |||||+++++++++----------------                                                
 |||||+++++++++----------------                                                
 |||||+++++++++++                                                              
 |||||+++++++++++                                                              
 |||||+++++++++++                                                              
 |||||+++++++++++                                                              
      +++++++++++                                                              
      +++++++++++                                                              
      +++++++++++                                                              
      +++++++++++                                                              
      +++++++++++
...
\end{verbatim}
    \vspace{-.3cm}
  \end{block}
\end{frame}

%24
\begin{frame}
  \begin{exampleblock}{\textattachfile{teglalap1.cpp}{teglalap1.cpp}}
    \scriptsize
    \vspace{-.2cm}
    \lstinputlisting[style=cpp,linerange={1-16},numbers=left,firstnumber=1]{teglalap1.cpp}
    \vspace{-.2cm}
  \end{exampleblock}
\end{frame}

%25
\begin{frame}
  \begin{exampleblock}{\textattachfile{teglalap1.cpp}{teglalap1.cpp}}
    \tiny
    \vspace{-.2cm}
    \lstinputlisting[style=cpp,linerange={48-72},numbers=left,firstnumber=48]{teglalap1.cpp}
    \vspace{-.2cm}
  \end{exampleblock}
\end{frame}

%26
\begin{frame}
  \begin{exampleblock}{\textattachfile{teglalap1.cpp}{teglalap1.cpp}}
    \lstinputlisting[style=cpp,linerange={38-47},numbers=left,firstnumber=38]{teglalap1.cpp}
  \end{exampleblock}
\end{frame}

%27
\begin{frame}
  \begin{exampleblock}{\textattachfile{teglalap1.cpp}{teglalap1.cpp}}
    \meret{7}
    \vspace{-.2cm}
    \lstinputlisting[style=cpp,linerange={18-37},numbers=left,firstnumber=18]{teglalap1.cpp}
    \vspace{-.2cm}
  \end{exampleblock}
\end{frame}

\end{document}