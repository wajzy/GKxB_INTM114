\documentclass[usenames,dvipsnames,aspectratio=169]{beamer}
\usepackage{../common/prg}

\title[7. előadás]{Programozás}
\subtitle{(GKxB\_INTM114)}

\begin{document}

%1
\begin{frame}[plain]
  \titlepage
  \logoalul
\end{frame}

%2
\section{Mutatók}
\subsection{Érték szerinti paraméter-átadás korlátai}
\begin{frame}
  Feladat:
  \begin{itemize}
    \item[] Készítsünk olyan függvényt, ami a két paraméterének értékét felcseréli!
  \end{itemize}
  Problémák:
  \begin{itemize}
    \item Érték szerinti paraméter-átadás
    \item Csak egy visszatérési értéke lehet a függvénynek
  \end{itemize}
  \begin{exampleblock}{\textattachfile{csere1.cpp}{csere1.cpp}}
    \scriptsize
    \lstinputlisting[style=cpp,linerange={1-6},numbers=left,firstnumber=1]{csere1.cpp}
  \end{exampleblock}
\end{frame}

%3
\begin{frame}[fragile]
  \small
  \begin{alertblock}{\textattachfile{csere1.cpp}{csere1.cpp} -- Első próbálkozás, \texttt{csere1}}
    \vspace{-.2cm}
    \scriptsize
    \lstinputlisting[style=cpp,linerange={8-12},numbers=left,firstnumber=8]{csere1.cpp}
    \vspace{-.2cm}
  \end{alertblock}
  \small
  \begin{alertblock}{\textattachfile{csere1.cpp}{csere1.cpp} -- A \texttt{main} részlete}
    \vspace{-.2cm}
    \scriptsize
    \lstinputlisting[style=cpp,linerange={27-30},numbers=left,firstnumber=27]{csere1.cpp}
    \vspace{-.2cm}
  \end{alertblock}
  \small
  \begin{block}{Kimenet részlete}
    \vspace{-.3cm}
    \scriptsize
    \begin{verbatim}
eredeti ertekek:        a = 1, b = 2
csere1 utan:            a = 1, b = 2
\end{verbatim}
    \vspace{-.3cm}
  \end{block}
\end{frame}

%4
\begin{frame}[fragile]
  \begin{alertblock}{\textattachfile{csere1.cpp}{csere1.cpp} -- Második próbálkozás, \texttt{csere2}}
    \vspace{-.2cm}
    \scriptsize
    \lstinputlisting[style=cpp,linerange={14-19},numbers=left,firstnumber=14]{csere1.cpp}
    \vspace{-.2cm}
  \end{alertblock}
  \begin{alertblock}{\textattachfile{csere1.cpp}{csere1.cpp} -- A \texttt{main} részlete}
    \vspace{-.2cm}
    \scriptsize
    \lstinputlisting[style=cpp,linerange={31-32},numbers=left,firstnumber=31]{csere1.cpp}
    \vspace{-.2cm}
  \end{alertblock}
  \begin{block}{Kimenet részlete}
    \vspace{-.3cm}
    \scriptsize
    \begin{verbatim}
csere2 utan:            a = 2, b = 1
\end{verbatim}
    \vspace{-.3cm}
  \end{block}
\end{frame}

%5
\subsection{Mutatók legfontosabb tulajdonságai, operátorok}
\begin{frame}[fragile]
  Mi az a \emph{mutató} (pointer), és mire használható?
  \begin{compactitem}
    \item Memóriacím tárolására használható típus
    \item Többféle típusa létezik, hogy kifejezze az ott tárolt adat típusát
    \item Technikai megvalósítása hasonlít az egész számokéhoz
    \item Mutatódefiníció: \emph{alaptípus}* \emph{azonosító};
  \end{compactitem}
  \begin{exampleblock}{Néhány lehetséges mutatódefiníció}
    \vspace{-.4cm}
    \small
    \begin{verbatim}
struct koordinata {
  int x, y;
};
/* ... */
char* pc;       // karaktert címző mutató
int* pi;        // egészet címző mutató
double* pd;     // valós számot c. m.
koordinata* pk; // struktúrát c. m.
void* pv;       // ismeretlen típusú adatot címző mutató
\end{verbatim}
    \vspace{-.4cm}
  \end{exampleblock}
\end{frame}

%6
\begin{frame}[fragile]
  \begin{exampleblock}{\textattachfile{cim.cpp}{cim.cpp} -- Memóriacím képzése az \& (címképző, address-of) operátorral lehetséges}
    \vspace{-.2cm}
    \scriptsize
    \lstinputlisting[style=cpp,linerange={5-12},numbers=left,firstnumber=5]{cim.cpp}
    \vspace{-.2cm}
  \end{exampleblock}
  \begin{block}{Kimenet}
    \vspace{-.3cm}
    \scriptsize
    \begin{verbatim}
Az i valtozo memoriacime: 0x7ffeb6837cdc
A k struktura memoriacime: 0x7ffeb6837ce0
k.x helye: 0x7ffeb6837ce0, k.y helye: 0x7ffeb6837ce4
A dt tomb memoriacime: 0x7ffeb6837cf0
dt[0] helye: 0x7ffeb6837cf0, dt[1] helye: 0x7ffeb6837cf8
\end{verbatim}
    \vspace{-.3cm}
  \end{block}
\end{frame}

%7
\begin{frame}[fragile]
  \begin{exampleblock}{\textattachfile{indirekcio.cpp}{indirekcio.cpp} -- Adott címen lévő érték elérése: * (indirekció, dereference) operátorral}
    \vspace{-.2cm}
    \footnotesize
    \lstinputlisting[style=cpp,numbers=left]{indirekcio.cpp}
    \vspace{-.2cm}
  \end{exampleblock}
  \begin{block}{Kimenet}
    \vspace{-.3cm}
    \footnotesize
    \begin{verbatim}
i = 5
\end{verbatim}
    \vspace{-.3cm}
  \end{block}
\end{frame}

%8
\begin{frame}[fragile]
  \begin{exampleblock}{\textattachfile{csere1.cpp}{csere1.cpp} -- Harmadik próbálkozás, \texttt{csere3}}
    \vspace{-.2cm}
    \scriptsize
    \lstinputlisting[style=cpp,linerange={21-25},numbers=left,firstnumber=21]{csere1.cpp}
    \vspace{-.2cm}
  \end{exampleblock}
  \begin{exampleblock}{\textattachfile{csere1.cpp}{csere1.cpp} -- A \texttt{main} részlete}
    \vspace{-.2cm}
    \scriptsize
    \lstinputlisting[style=cpp,linerange={33-35},numbers=left,firstnumber=33]{csere1.cpp}
    \vspace{-.2cm}
  \end{exampleblock}
  \begin{block}{Kimenet részlete}
    \vspace{-.3cm}
    \scriptsize
    \begin{verbatim}
csere2 utan:            a = 2, b = 1
csere3 utan:            a = 1, b = 2
\end{verbatim}
    \vspace{-.2cm}
  \end{block}
\end{frame}

%9
\begin{frame}
  \footnotesize
  \begin{exampleblock}{\textattachfile{teglalap2.cpp}{teglalap2.cpp} Téglalapok rajzolása, \texttt{bekerBFX}: Akarja folytatni a bevitelt? Ha igen, 
mi a koordináta?}
    \meret{7}
    \vspace{-.2cm}
    \lstinputlisting[style=cpp,linerange={38-56},numbers=left,firstnumber=38]{teglalap2.cpp}
    \vspace{-.2cm}
  \end{exampleblock}
\end{frame}

%10
\begin{frame}
  \small
  \begin{exampleblock}{\textattachfile{teglalap2.cpp}{teglalap2.cpp}}
    \vspace{-.2cm}
    \scriptsize
    \lstinputlisting[style=cpp,linerange={58-75},numbers=left,firstnumber=58]{teglalap2.cpp}
    \vspace{-.2cm}
  \end{exampleblock}
\end{frame}

%11
\subsection{További tudnivalók a mutatókkal kapcsolatban}
\begin{frame}
  Néhány további tudnivaló mutatókkal kapcsolatban
  \begin{itemize}
    \item \kiemel{Vigyázz!} \texttt{i} egész, \texttt{pi1} és \texttt{pi2} viszont egészet címző mutatók!
    \item A \kiemel{*} körül tetszőleges számú szóköz elhelyezhető
    \item Mutató is kaphat inicializálással kezdőértéket
  \end{itemize}
  \begin{exampleblock}{\textattachfile{mutatok1.cpp}{mutatok1.cpp}}
    \lstinputlisting[style=cpp,linerange={5-8},numbers=left,firstnumber=5]{mutatok1.cpp}
  \end{exampleblock}
\end{frame}

%12
\begin{frame}
  \begin{itemize}
    \item Ha valaminek nincs memóriacíme, akkor az \& operátor sem tudja előállítani
  \end{itemize}
  \begin{alertblock}{\textattachfile{mutatok1.cpp}{mutatok1.cpp}}
    \small
    \lstinputlisting[style=cpp,linerange={9-11},numbers=left,firstnumber=9]{mutatok1.cpp}
  \end{alertblock}
  \begin{itemize}
    \item Értékadás általában csak azonos típusú mutatók között lehetséges
  \end{itemize}
  \begin{alertblock}{\textattachfile{mutatok1.cpp}{mutatok1.cpp}}
    \small
    \lstinputlisting[style=cpp,linerange={12-13},numbers=left,firstnumber=12]{mutatok1.cpp}
  \end{alertblock}
\end{frame}

%13
\begin{frame}
  \begin{compactitem}
    \small
    \item Kivétel: \texttt{void*} bármilyen más mutató értékét felveheti ($\approx$ típusinformáció eldobása)
  \end{compactitem}
  \begin{exampleblock}{\textattachfile{mutatok1.cpp}{mutatok1.cpp}}
    \vspace{-.2cm}
    \small
    \lstinputlisting[style=cpp,linerange={14-15},numbers=left,firstnumber=14]{mutatok1.cpp}
    \vspace{-.2cm}
  \end{exampleblock}
  \begin{compactitem}
    \small
    \item Fordítva már nem megy: nem lehetünk benne biztosak, hogy azon a címen milyen típusú adat van
    \item Explicit típuskonverzióval persze rávehető a fordító a műveletre, de vajon van értelme?
  \end{compactitem}
  \begin{alertblock}{\textattachfile{mutatok1.cpp}{mutatok1.cpp}}
    \vspace{-.2cm}
    \small
    \lstinputlisting[style=cpp,linerange={16-18},numbers=left,firstnumber=16]{mutatok1.cpp}
    \vspace{-.2cm}
  \end{alertblock}
\end{frame}

%14
\begin{frame}
  \begin{itemize}
    \small
    \item Nem tudni, hogyan kell megjeleníteni az ismeretlen típusú adatot
  \end{itemize}
  \begin{alertblock}{\textattachfile{mutatok1.cpp}{mutatok1.cpp}}
    \vspace{-.2cm}
    \small
    \lstinputlisting[style=cpp,linerange={19-20},numbers=left,firstnumber=19]{mutatok1.cpp}
    \vspace{-.2cm}
  \end{alertblock}
  \begin{itemize}
    \small
    \item A \texttt{NULL} / \texttt{nullptr} speciális memóriacím: semmilyen adatot nem tárolnak ott, és
    \item hiba, vagy valami hiányának jelzésére használják,
    \item bármilyen típusú mutatóhoz hozzárendelhető érték
    \item \texttt{NULL}: a 0 értékhez készített makró (C örökség, elavult), \texttt{nullptr}: mindenképpen mutató
  \end{itemize}
  \begin{exampleblock}{\textattachfile{mutatok1.cpp}{mutatok1.cpp}}
    \vspace{-.2cm}
    \footnotesize
    \lstinputlisting[style=cpp,linerange={22-22},numbers=left,firstnumber=22]{mutatok1.cpp}
    \vspace{-.2cm}
  \end{exampleblock}
\end{frame}

%15
\subsection{Mintapélda: dátumok kezelése}
\begin{frame}
  Probléma:
  \begin{itemize}
    \item[] struktúrák általában nagyok, függvényhívásnál a paraméter átadás a másolás miatt időigényes
  \end{itemize}
  \vfill
  Megoldás:
  \begin{itemize}
    \item adjuk át a struktúra címét!
    \item \kiemel{Veszély!} Ha a \emph{hívott} fv. módosítja a paramétert, annak a \emph{hívó} függvényben is lesz hatása!
    \item Ha a \emph{hívott} függvénynek nem célja módosítani a paramétert: \kiemel{\texttt{const}} csak olvashatóvá teszi azt
(bármilyen más változónál is használható \emph{típusmódosító})
    \item Indirekció + tagelérés: \kiemel{\texttt{->}} operátorral, pl. \texttt{(*d).nap} $\equiv$ \texttt{d->nap}
  \end{itemize}
\end{frame}

%16
\begin{frame}
  \begin{exampleblock}{\textattachfile{naptar2.cpp}{naptar2.cpp}}
    \footnotesize
    \lstinputlisting[style=cpp,linerange={23-36},numbers=left,firstnumber=23]{naptar2.cpp}
  \end{exampleblock}
\end{frame}

%17
\begin{frame}
  \small
  \begin{exampleblock}{\textattachfile{naptar2.cpp}{naptar2.cpp}}
    \vspace{-.2cm}
    \footnotesize
    \lstinputlisting[style=cpp,linerange={77-91},numbers=left,firstnumber=77]{naptar2.cpp}
    \vspace{-.2cm}
  \end{exampleblock}
\end{frame}

%18
\subsection{Mutatók és tömbök, mutatóaritmetika}
\begin{frame}
  \small
  \begin{exampleblock}{\textattachfile{buborek2.cpp}{buborek2.cpp}}
    \vspace{-.2cm}
    \small
    \lstinputlisting[style=cpp,linerange={1-14},numbers=left,firstnumber=1]{buborek2.cpp}
    \vspace{-.2cm}
  \end{exampleblock}
\end{frame}

%19
\begin{frame}[fragile]
  \small
  \begin{exampleblock}{\textattachfile{buborek2.cpp}{buborek2.cpp}}
    \vspace{-.2cm}
    \footnotesize
    \lstinputlisting[style=cpp,linerange={16-26},numbers=left,firstnumber=16]{buborek2.cpp}
    \vspace{-.2cm}
  \end{exampleblock}
  \begin{block}{Kimenet}
    \vspace{-.4cm}
    \footnotesize
    \begin{verbatim}
Rendezes utan:
-4      3       3       4       7       12      54      56
\end{verbatim}
    \vspace{-.3cm}
  \end{block}
\end{frame}

%20
\begin{frame}
  Újdonságok:
  \begin{itemize}
    \item Tömb elemszámát nem \emph{kell} megadni a formális paraméterlistán (de a fv.-nek valahonnan tudnia kell, hány tömbelemet kell rendezni)
    \item A \emph{hívott} függvény \kiemel{módosította} a paraméter tömböt!
  \end{itemize}
  Magyarázat:
  \begin{itemize}
    \item A tömbök általában nagyok $\to$ \kiemel{mindig} a címet adják át!
    \item A tömbök azonosítója egy \emph{konstans mutató} (a mutatót nem, de a mutatott helyen lévő értéket lehet módosítani), pl.\\ 
    \texttt{int t[]} $\equiv$ \texttt{int* const t}
    \item A tömb tartalma csak olvashatóvá tehető: \\ 
    \texttt{const int t[]} $\equiv$ \texttt{const int* const t}
    \item Hátulról előre olvasva: \emph{\texttt{t}} egy \emph{\texttt{const}}-ans mutató (\emph{\texttt{*}}), ami olyan \emph{\texttt{int}}-et címez ami \emph{\texttt{const}}-ans.
  \end{itemize}
\end{frame}

%21
\begin{frame}
  \begin{exampleblock}{\textattachfile{buborek3.cpp}{buborek3.cpp}}
    \vspace{-.2cm}
    \footnotesize
    \lstinputlisting[style=cpp,linerange={16-30},numbers=left,firstnumber=16]{buborek3.cpp}
    \vspace{-.2cm}
  \end{exampleblock}
\end{frame}

%22
\begin{frame}
  Mutatóaritmetika: hasonlóan végezhető művelet mutatókkal, mint egészekkel:
  \begin{itemize}
    \item Mutató növelhető, csökkenthető $\to$ a tényleges cím a mutatott adat méretének többszörösével változik
    \item Mutatók összehasonlíthatóak (relációk)
    \item tömbelemCíme = tömbKezdőcíme + index*sizeof(tömbelemTípusa)
    \item \texttt{tomb[index]} $\equiv$ \texttt{*(tomb + index)}
    \item A \texttt{void*} mutató kivételes: a mutatott elem mérete ismeretlen
    \item Azonos tömb elemeit címző mutatók különbsége képezhető
  \end{itemize}
\end{frame}

%23
\begin{frame}[fragile]
  \begin{exampleblock}{\textattachfile{mutatok2.cpp}{mutatok2.cpp}}
    \vspace{-.2cm}
    \footnotesize
    \lstinputlisting[style=cpp,linerange={5-12},numbers=left,firstnumber=5]{mutatok2.cpp}
    \vspace{-.2cm}
  \end{exampleblock}
  \begin{block}{Kimenet}
    \vspace{-.3cm}
    \begin{verbatim}
Elso elem erteke (cime):        100 (0x7ffc05f63510)
Masodik elem erteke (cime):     200 (0x7ffc05f63514)
Harmadik elem erteke (cime):    300 (0x7ffc05f63518)     
    \end{verbatim}
    \vspace{-.6cm}
  \end{block}
\end{frame}

%24
\begin{frame}
  \begin{exampleblock}{\textattachfile{buborek4.cpp}{buborek4.cpp}}
    \lstinputlisting[style=cpp,linerange={16-21},numbers=left,firstnumber=16]{buborek4.cpp}
  \end{exampleblock}
\end{frame}

%25
\section{Véletlenszám generálás}
\subsection{A \emph{C} nyelv függvénykönytárának szolgáltatásai}
\begin{frame}
  A C nyelvből örökölt megoldás:
  \begin{itemize}
    \item \hiv{\href{https://en.wikipedia.org/wiki/Pseudorandom_number_generator}{Álvéletlen számok}} előállítása 
(PseudoRandom Number Generator, PRNG)
    \item Szükséges fejfájl: \texttt{cstdlib} vagy \texttt{stdlib.h}
    \item Kezdőérték: \texttt{void srand(unsigned int seed);}, ahol \texttt{seed} a kezdőérték
    \item Véletlen számok: $0 \leq$ \texttt{int rand(void);} $\leq$ \texttt{RAND\_MAX}
  \end{itemize}
  \vfill
  Példák:
   \begin{itemize}
     \item \texttt{x = (double)rand()/RAND\_MAX} ahol $\left\{x|x\in\mathbb{R}, 0 \leq x \leq 1 \right\}$
     \item \texttt{x = MIN + rand()\%(MAX-MIN+1)} ahol $\left\{x|x\in\mathbb{Z}, MIN \leq x \leq MAX \right\}$
   \end{itemize}
\end{frame}

%26
\begin{frame}
  Probléma: azonos \texttt{seed} $\to$ azonos számsorozatok\\
  Megoldás:
  \begin{itemize}
    \item \texttt{seed} minden programindításnál más legyen $\to$ pontos idő
    \item Szükséges fejfájl: \texttt{ctime} vagy \texttt{time.h}
    \item \texttt{time\_t time(time\_t *t);}
    \item V.t. érték: \texttt{time\_t} (\texttt{long}) típusban az 1970-01-01 00:00:00 +0000 (UTC) (\hiv{\href{https://hu.wikipedia.org/wiki/Unix-id\%C5\%91}{Unix-idő}}, epoch)
óta eltelt \kiemel{másodpercek (!)} száma, amit \texttt{t} címen is eltárol, ha az nem \texttt{nullptr}
  \end{itemize}
  A C++ nyelv \hiv{\href{http://www.cplusplus.com/reference/random/}{sokkal kifinomultabb képességekkel}} rendelkezik.\\
  \hiv{\href{http://www.cplusplus.com/faq/beginners/random-numbers/}{További tudnivalók}} a véletlenszám generálással 
kapcsolatban.
\end{frame}

%27
\begin{frame}
  \small
  \begin{exampleblock}{\textattachfile{tipp.cpp}{tipp.cpp}}
    \vspace{-.2cm}
    \scriptsize
    \lstinputlisting[style=cpp,numbers=left]{tipp.cpp}
    \vspace{-.2cm}
  \end{exampleblock}
\end{frame}

\end{document}