\documentclass[usenames,dvipsnames,aspectratio=169]{beamer}
\usepackage{../common/prg}

\title[3. előadás]{Programozás}
\subtitle{(GKxB\_INTM114)}

\begin{document}

%1
\begin{frame}[plain]
  \titlepage
  \logoalul
\end{frame}

%2
\section{Tömbök}
\subsection{Kódismétlés csökkentése tömbökkel}
\begin{frame}
  \begin{exampleblock}{\textattachfile{haromszog1.cpp}{haromszog1.cpp} Háromszög szerkeszthetősége; az oldalhossz beolvasása 3x ismétlődik!}
    \tiny
    \vspace{-.2cm}
    \lstinputlisting[style=cpp,numbers=left]{haromszog1.cpp}
    \vspace{-.2cm}
  \end{exampleblock}
\end{frame}

%3
\begin{frame}
  \begin{exampleblock}{\textattachfile{haromszog3.cpp}{haromszog3.cpp}}
    \tiny
    \vspace{-.2cm}
    \lstinputlisting[style=cpp,numbers=left]{haromszog3.cpp}
    \vspace{-.2cm}
  \end{exampleblock}
\end{frame}

%4
\begin{frame}
  Tömb definíció
  \begin{itemize}
    \item \emph{típus tömbazonosító[méret];}
    \item pl. \texttt{int ot[3];}
    \item \emph{méret} pozitív, egész értékű \emph{állandó kifejezés}
    \item \emph{állandó kifejezés} értéke fordítási időben kiszámítható
  \end{itemize}
  Tömb tárigénye
  \begin{itemize}
    \item[] \emph{sizeof(tömbazonosító) $\equiv$ méret*sizeof(típus)}
  \end{itemize}
  Tömbelemek (indexes változó) elérése
  \begin{itemize}
    \item \emph{tömbazonosító[index]}
    \item 0 $\leq$ \emph{index} $\leq$ \emph{méret}$-$1
  \end{itemize}
\end{frame}

%5
\begin{frame}
  Oldal nevének előállítása
  \begin{itemize}
    \item Kihasználjuk, hogy az \hiv{\href{https://www.ascii-code.com/}{ASCII kódok}} a betűk abc-beli sorrendjének megfelelően növekednek ('A' == 65, 'B' == 66, 
\dots, 'Z' == 90)
    \item Hasonló a helyzet a számjegyekkel is ('0' == 48, '1' == 49, \dots, '9' == 57)
    \item Számjegy $\to$ ASCII kód: \texttt{'0' + szamjegy}
    \item ASCII kód $\to$ Számjegy: \texttt{karakter - '0'}
    \item Betűk is hasonlóan kezelhetők
  \end{itemize}
\end{frame}

%6
\subsection{Számlálás tömbbel}
\begin{frame}
  \begin{exampleblock}{\textattachfile{szamlalo1.cpp}{szamlalo1.cpp} Számjegy karakterek számlálása, 1/2}
    \tiny
    \vspace{-.2cm}
    \lstinputlisting[style=cpp,lastline=24,numbers=left]{szamlalo1.cpp}
    \vspace{-.2cm}
  \end{exampleblock}
\end{frame}

%7
\begin{frame}
  \begin{exampleblock}{\textattachfile{szamlalo1.cpp}{szamlalo1.cpp} 2/2}
    \tiny
    \lstinputlisting[style=cpp,firstline=25,numbers=left,firstnumber=25]{szamlalo1.cpp}
  \end{exampleblock}
  \vfill
  Nyilvánvalóan szükségünk van egy tömbre!
\end{frame}

%8
\begin{frame}
  \begin{exampleblock}{\textattachfile{szamlalo2.cpp}{szamlalo2.cpp} 1/2}
    \footnotesize
    \vspace{-.2cm}
    \lstinputlisting[style=cpp,lastline=15,numbers=left]{szamlalo2.cpp}
    \vspace{-.2cm}
  \end{exampleblock}
\end{frame}

%9
\begin{frame}
  \begin{exampleblock}{\textattachfile{szamlalo2.cpp}{szamlalo2.cpp} 2/2}
    \scriptsize
    \lstinputlisting[style=cpp,firstline=16,numbers=left,firstnumber=16]{szamlalo2.cpp}
  \end{exampleblock}
\end{frame}

%10
\begin{frame}
  Tömbelemek, mint számlálók
  \begin{itemize}
    \item[] Az \kiemel{i} számjegy darabszámát \kiemel{szamjegy[i]} tárolja! (Azaz pl. 0-ból szamjegy[0], 1-ből szamjegy[1], stb. érkezett.)
  \end{itemize}
  \vfill
  Tömbök inicializálása
  \begin{itemize}
    \item \emph{típus tömbazonosító[<méret>]<={inicializátorlista}>;}
    \item Ha \emph{inicializátorlista} elemszáma $<$ \emph{méret} $\to$ további elemek nullázódnak
    \item Ha \emph{inicializátorlista} elemszáma $>$ \emph{méret} $\to$ hiba!
    \item Ha \emph{méret}-et nem specifikálták, a fordító megállapítja \emph{inicializátorlista} elemszámából
    \item De a \emph{méret} és az \emph{inicializátorlista} közül legalább az egyiknek léteznie kell!
  \end{itemize}
\end{frame}

%11
\begin{frame}
  \begin{exampleblock}{\textattachfile{szamlalo3.cpp}{szamlalo3.cpp}}
    \tiny
    \vspace{-.2cm}
    \lstinputlisting[style=cpp,numbers=left]{szamlalo3.cpp}
    \vspace{-.2cm}
  \end{exampleblock}
\end{frame}

%12
\section{Néhány elemi, tömböt használó algoritmus}
\subsection{Számok kiírása fordított sorrendben}
\begin{frame}
  \begin{exampleblock}{\textattachfile{forditva1.cpp}{forditva1.cpp}}
    \meret{7}
    \lstinputlisting[style=cpp,numbers=left]{forditva1.cpp}
  \end{exampleblock}
\end{frame}

%13
\subsection{Bináris keresés}
\begin{frame}
  \begin{exampleblock}{\textattachfile{binker.cpp}{binker1.cpp} \kiemelN{\href{https://hu.wikipedia.org/wiki/Bin\%C3\%A1ris\_keres\%C3\%A9s}{Bináris keresés}}, 1/4}
    \tiny
    \vspace{-.2cm}
    \lstinputlisting[style=cpp,numbers=left]{binker1.cpp}
    \vspace{-.2cm}
  \end{exampleblock}
\end{frame}

%14
\begin{frame}
  \begin{exampleblock}{\textattachfile{binker.cpp}{binker2.cpp} \kiemelN{\href{https://hu.wikipedia.org/wiki/Bin\%C3\%A1ris\_keres\%C3\%A9s}{Bináris keresés}}, 2/4}
    \tiny
    \vspace{-.2cm}
    \lstinputlisting[style=cpp,numbers=left]{binker2.cpp}
    \vspace{-.2cm}
  \end{exampleblock}
\end{frame}

%15
\begin{frame}
  \begin{exampleblock}{\textattachfile{binker.cpp}{binker3.cpp} \kiemelN{\href{https://hu.wikipedia.org/wiki/Bin\%C3\%A1ris\_keres\%C3\%A9s}{Bináris keresés}}, 3/4}
    \tiny
    \vspace{-.2cm}
    \lstinputlisting[style=cpp,numbers=left]{binker3.cpp}
    \vspace{-.2cm}
  \end{exampleblock}
\end{frame}

%16
\begin{frame}
  \begin{exampleblock}{\textattachfile{binker.cpp}{binker4.cpp} \kiemelN{\href{https://hu.wikipedia.org/wiki/Bin\%C3\%A1ris\_keres\%C3\%A9s}{Bináris keresés}}, 4/4}
    \tiny
    \vspace{-.2cm}
    \lstinputlisting[style=cpp,numbers=left]{binker4.cpp}
    \vspace{-.2cm}
  \end{exampleblock}
\end{frame}

%17
\subsection{Buborék rendezés}
\begin{frame}
  \begin{exampleblock}{\textattachfile{buborek.cpp}{buborek.cpp} \kiemelN{\href{https://hu.wikipedia.org/wiki/Bubor\%C3\%A9krendez\%C3\%A9s}{Buburékrendezés}}, %
    \kiemelN{\href{https://www.baeldung.com/java-swap-two-variables}{elemcsere segédváltozó nélkül}}}
    \tiny
    \vspace{-.2cm}
    \lstinputlisting[style=cpp,numbers=left]{buborek.cpp}
    \vspace{-.2cm}
  \end{exampleblock}
\end{frame}

%18
\section{C++ karakterláncok, karakterek osztályozása és konverziója}
\subsection{Az \texttt{std::string} típus}
\begin{frame}
  \begin{exampleblock}{\textattachfile{string1.cpp}{string1.cpp} string demo}
    \scriptsize
    \vspace{-.2cm}
    \lstinputlisting[style=cpp,numbers=left]{string1.cpp}
    \vspace{-.2cm}
  \end{exampleblock}
\end{frame}

%19
\begin{frame}
  \begin{columns}[c]
    \column{.5\textwidth}
      \begin{exampleblock}{\textattachfile{string2.cpp}{string2.cpp} Iterálás a karakterláncon}
        \scriptsize
        \vspace{-.2cm}
        \lstinputlisting[style=cpp,numbers=left]{string2.cpp}
        \vspace{-.2cm}
      \end{exampleblock}
    \column{.4\textwidth}
      A \hiv{\href{https://cplusplus.com/reference/cstddef/size_t/}{\texttt{size\_t}}} típus pl. a \texttt{sizeof} operátor és a \hiv{\href{https://cplusplus.com/reference/string/string/length/}{\texttt{std::string::length()}}} által visszaadott érték típusa; előjel nélküli, valaminek a méretét vagy darabszámát fejezi ki.
\end{columns}
  \begin{block}{Fordító kimenete}
    \scriptsize
    string2.cpp:7:17: warning: comparison of integer expressions of different signedness: 'int' and 'std::\_\_cxx11::basic\_string<char>::size\_type' {aka 'long unsigned int'} [-Wsign-compare]
  \end{block}
\end{frame}

%20
\subsection{Konvertálás kettesből tízes számrendszerbe}
\begin{frame}
  \begin{exampleblock}{\textattachfile{kettes1.cpp}{kettes1.cpp} 2 $\to$ 10}
    \footnotesize
    \vspace{-.2cm}
    \lstinputlisting[style=cpp,numbers=left]{kettes1.cpp}
    \vspace{-.2cm}
  \end{exampleblock}
\end{frame}

%21
\subsection{Konvertálás tízesből kettes számrendszerbe}
\begin{frame}
  \begin{exampleblock}{\textattachfile{kettes2.cpp}{kettes2.cpp} 10 $\to$ 2}
    \small
    \lstinputlisting[style=cpp,numbers=left]{kettes2.cpp}
  \end{exampleblock}
\end{frame}

%22
\subsection{Neptun kód ellenőrzés}
\begin{frame}
  \begin{exampleblock}{\textattachfile{neptun1.cpp}{neptun1.cpp}}
  \tiny
  \vspace{-.2cm}
  \lstinputlisting[style=cpp,numbers=left]{neptun1.cpp}
  \vspace{-.2cm}
\end{exampleblock}
\end{frame}

%23
\begin{frame}
  \scriptsize
  \begin{exampleblock}{\textattachfile{neptun2.cpp}{neptun2.cpp}}
  \meret{7}
  \vspace{-.2cm}
  \lstinputlisting[style=cpp,numbers=left]{neptun2.cpp}
  \vspace{-.2cm}
\end{exampleblock}
\end{frame}

%24
\begin{frame}
  Karakterek osztályozása, átalakítása
  \begin{itemize}
    \item \hiv{\href{https://en.cppreference.com/w/cpp/header/cctype}{\texttt{cctype}}} vagy \texttt{ctype.h} beszerkesztése szükséges
    \item Függvények vagy makrók (előfeldolgozó)
    \item Paraméter típusa \texttt{int}, de az értéknek \texttt{unsigned char}-ral ábrázolhatónak, vagy \texttt{EOF}-nak kell lennie
    \item Visszatérési érték \texttt{int}, karakterosztályozó rutinoknál logikai értékként kezelendő
  \end{itemize}
  \begin{columns}[T]
    \column{.4\textwidth}
      \meret{9}
      \begin{tabular}{ll}
        Fv./makró név & Funkció\\ \hline
        \texttt{islower(c)} & \texttt{c} kisbetű?\\
        \texttt{isupper(c)} & \texttt{c} nagybetű?\\
        \texttt{isalpha(c)} & \texttt{c} betű?\\
        \texttt{isdigit(c)} & \texttt{c} számjegy?\\
        \texttt{isalnum(c)} & \texttt{c} alfanumerikus?\\
      \end{tabular}
    \column{.55\textwidth}
      \meret{9}
      \begin{tabular}{ll}
        Fv./makró név & Funkció\\ \hline
        \texttt{isxdigit(c)} & \texttt{c} hexadecimális számjegy?\\
        \texttt{isspace(c)} & \texttt{c} fehér karakter?\\
        \texttt{isprint(c)} & \texttt{c} nyomtatható?\\
        \texttt{tolower(c)} & \texttt{c} kisbetűs alakja, ha \texttt{c} nagybetű\\
        \texttt{toupper(c)} & \texttt{c} nagybetűs alakja, ha \texttt{c} kisbetű
      \end{tabular}
  \end{columns}
\end{frame}

%25
\begin{frame}
  \begin{exampleblock}{\textattachfile{neptun3.cpp}{neptun3.cpp}}
  \meret{7}
  \vspace{-.2cm}
  \lstinputlisting[style=cpp,numbers=left]{neptun3.cpp}
  \vspace{-.2cm}
\end{exampleblock}
\end{frame}

\end{document}